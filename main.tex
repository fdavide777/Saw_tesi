\documentclass[12pt]{article}
\usepackage[a4paper, margin=1in, headheight=57pt]{geometry}
\usepackage{amssymb}
\usepackage{braket}

\begin{document}
\date{}
\author{Ferraro Davide}
\title{\textbf{Polimeri e Self Avoiding Walks}}
\maketitle
\begin{center}
Laurea Triennale in Fisica \linebreak
Università di Pisa, Dipartimento di Fisica
\end{center}
Un \textit{polimero} è una molecola composta da una catena di \textit{monomeri} (gruppi di atomi) tenuti assieme da legami chimici, tipicamente covalenti. Considerando le interazioni tra uno degli $n$ monomeri ed un numero $g$ di suoi vicini, possiamo definire una catena composta da $n/g$ sotto unità, che hanno il vantaggio di essere indipendenti tra loro. Ciò significa che, se la catena è abbastanza lunga, le interazioni tra i singoli monomeri sono trascurabili se siamo interessati a proprietà globali, e possiamo considerare il polimero come una catena fatta da sotto-elementi a distanza fissata.
I polimeri, specialmente quelli \textit{lineari} (nei quali ogni monomero fa due legami), possono essere molto grandi; alcuni di essi sono composti da più di $10^5$ monomeri. Inoltre, il volume occupato da ciascuno dei monomeri, detto \textit{volume escluso}, non può essere occupato da nessun altro monomero.

Ne consegue che un polimero lineare è ben modellato da un \textit{self avoiding random walk} (o SAW), ossia un cammino randomico (che rappresenta la catena) che non interseca mai sè stesso (che rappresenta il volume escluso). Il numero di configurazioni che può adottare una catena da $n$ monomeri, oppure la distanza quadratica media tra i suoi estremi, sono domande che possono essere traslate nello studio dei SAW. Nella nostra ipotesi di $n$ grande, un polimero è modellato da un SAW che vive in un reticolo di interi. In particolare, il reticolo è in $\mathbb{Z}^3$ se il polimero è libero in soluzione, oppure in $\mathbb{Z}^2$ se il polimero è trattenuto su una qualche superficie (\textit{adsorbimento}). Nel mio studio mi sono concentrato su quest'ultimo caso, ed in particolare su polimeri lineari che quindi sono modellati da cammini su un reticolo quadrato.

Nonostante la semplice definizione, le proprietà matematiche dei SAW sono complesse, e sia il numero possibile di configurazioni $c_n$ che la distanza quadratica media $\braket{R_n^2}$ tra gli estremi di un cammino sono descritte solo da relazioni congetturate. In particolare, si crede esista un esponente critico $\nu$ ed una costante $A$ tale che $\braket{R^2_n}\sim AN^{2\nu}$, dove $A,\nu$ dipendono dalla dimensione in cui vive il reticolo ma non dalla sua geometria. In due dimensioni il valore dell'esponente $\nu$ si suppone sia pari a $3/4$.

I metodi Monte Carlo sono storicamente stati essenziali per studiare le proprietà dei SAW. Nel mio progetto di tesi ho utilizzato 3 diversi metodi per generare catene di lunghezza $n$ e stimarne il raggio quadratico medio $\braket{R_n^2}$: il \textit{simple sampling}, l'algoritmo di \textit{dimerizzazione} e l'algoritmo \textit{pivot}. Il simple sampling è un metodo \textit{statico}, che genera cammini di $n$ passi muovendosi casualmente sul reticolo, e scartando il risultato se il cammino si autointerseca; è estremamente inefficiente per $n$ grandi ($n\gtrsim 50$). L'algoritmo di dimerizzazione è anch'esso statico, e genera cammini lunghi combinando ricorsivamente cammini più corti. L'algoritmo pivot, invece, è un processo \textit{dinamico} e \textit{Markoviano}. Prende in input un cammino di $n$ passi e applica una qualche trasformazione ad un segmento del cammino; il nuovo cammino viene scartato se si verifica auto-intersecazione. Nonostante l'efficienza per $n$ sia minore rispetto ad altri processi dinamici grandi, l'algoritmo pivot garantisce una convergenza più veloce al valore esatto di $\braket{R_n^2}$.






%Il simple sampling, come suggerisce il nome, è un metodo intuitivo e facilmente implementabile. I passi del cammino sono generati randomicamente, con l'unica restrizione che ogni passo "ricordi" il passo precedente, in modo da evitare di tornare su sè stesso. Il simple sampling è un metodo \textit{statico}, ossia genera una sequenza di SAW indipendenti tra loro. L'efficienza dell'algoritmo descresce rapidamente con il crescere di $N$; per catene lunghe ($N\gtrsim70$) diventa proibitivo. L'algoritmo di dimerizzazione è essenzialmente un processo ricorsivo. L'idea è che se desideriamo generare una cammino di $N$ passi, allora possiamo generare due cammini indipendenti di $N/2$ passi e successivamente tentare di concatenarli. Per generare tali cammini da $N/2$ passi possiamo usare 4 cammini da $N/4$ passi e così via. La ricorsione si ferma dopo $k$ livelli se si ha un metodo veloce per generare SAW di $N_0=N/2^k$ passi; nel mio caso ho utilizzato il simple sampling con $N_0=10$. Anche questo è un metodo statico. L'algoritmo pivot, genera cammini di lunghezza $N$ a partire

\end{document}